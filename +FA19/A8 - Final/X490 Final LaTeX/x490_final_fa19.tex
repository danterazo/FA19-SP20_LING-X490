%% File acl2018.tex
%
%% Based on the style files for ACL-2017, with some changes, which were, in turn,
%% Based on the style files for ACL-2015, with some improvements
%%  taken from the NAACL-2016 style
%% Based on the style files for ACL-2014, which were, in turn,
%% based on ACL-2013, ACL-2012, ACL-2011, ACL-2010, ACL-IJCNLP-2009,
%% EACL-2009, IJCNLP-2008...
%% Based on the style files for EACL 2006 by 
%%e.agirre@ehu.es or Sergi.Balari@uab.es
%% and that of ACL 08 by Joakim Nivre and Noah Smith

\documentclass[11pt,a4paper]{article}
\usepackage[hyperref]{acl2018}
\usepackage{times,latexsym,url}

\aclfinalcopy % Uncomment this line for the final submission
%\def\aclpaperid{***} %  Enter the acl Paper ID here

%\setlength\titlebox{5cm}
% You can expand the titlebox if you need extra space
% to show all the authors. Please do not make the titlebox
% smaller than 5cm (the original size); we will check this
% in the camera-ready version and ask you to change it back.

\newcommand\BibTeX{B{\sc ib}\TeX}

\title{Biased vs. Random Sampling for Abusive Language Detection}

\author{Dante Razo \\
	Indiana University, Bloomington, IN \\
	Department of Computational Linguistics \\
	{\tt drazo@indiana.edu} \\}
\date{12/14/2019}

\begin{document}
\maketitle
%-----------------------------------------------------------------------
%	IDEA: Biased (Boosted) vs. Random Sampling
%-----------------------------------------------------------------------
% Resample on Kaggle since its huge. 
% 1. Boost by filtering on a topic and creating a list of words/hashtags (i.e. soccer: ["goalie", "manchester"]). Compare to random sampling.
% 3. Get abusive words list from lexicon-of-abusive-words repo
% 	- if 1+ words appear, then it's explicit; otherwise, implicit

% REMEMBER: it's okay to fail in research
% ALSO REMEMBER: this only needs to be 1pg. Doesn't need to be long

%-----------------------------------------------------------------------
%	PAPER
%-----------------------------------------------------------------------
\begin{abstract}
  This document contains the instructions for preparing a camera-ready
  manuscript for the proceedings of ACL 2018. The document itself
  conforms to its own specifications, and is therefore an example of
  what your manuscript should look like. These instructions should be
  used for both papers submitted for review and for final versions of
  accepted papers.  Authors are asked to conform to all the directions
  reported in this document.
\end{abstract}

\section{Introduction} % Research Question
% TODO: rewrite it
% Alternatively: Depending on what sampling you use, it makes the data more explicit or implicit

\section{Sampling Experiment}

\section{Bias Paper Overview}
% TODO: Wiegand paper

\subsection{Kaggle}

\subsubsection{SVM}

\subsection{Founta}

\subsubsection{SVM}
\subsubsection{Random Forest}
\subsubsection{Random Forest with GridSearch}

\subsection{Kumar}

\subsubsection{SVM}
\subsubsection{Random Forest}
\subsubsection{Random Forest with GridSearch}

% include your own bib file like this:
%\bibliographystyle{acl}
%\bibliography{acl2018}
%\bibliography{x490bib}
\bibliographystyle{acl_natbib}

\end{document}
